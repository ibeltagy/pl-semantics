% RECOMMENDED %%%%%%%%%%%%%%%%%%%%%%%%%%%%%%%%%%%%%%%%%%%%%%%%%%%
\documentclass[graybox]{styles/svmult}

% choose options for [] as required from the list
% in the Reference Guide

\usepackage{mathptmx}       % selects Times Roman as basic font
\usepackage{helvet}         % selects Helvetica as sans-serif font
\usepackage{courier}        % selects Courier as typewriter font
\usepackage{type1cm}        % activate if the above 3 fonts are
                            % not available on your system
%
\usepackage{makeidx}         % allows index generation
\usepackage{graphicx}        % standard LaTeX graphics tool
                             % when including figure files
\usepackage{multicol}        % used for the two-column index
\usepackage[bottom]{footmisc}% places footnotes at page bottom

% see the list of further useful packages
% in the Reference Guide

\makeindex             % used for the subject index
                       % please use the style svind.ist with
                       % your makeindex program





%\usepackage{url}       %%% for including URLs
\usepackage{natbib}
%\usepackage[margin=25mm]{geometry}


\usepackage{wrapfig}
\usepackage{subfig}
\usepackage{amsmath,amssymb}
%\usepackage{styles/covington}


\newcommand{\dhgdrs}[3]
{
    {
    \begin{tabular}{rl}
    \\
	\hspace{-100ex}
    #1 
    \hspace{-3ex}
	\vspace{-5ex}
    \\ & 
    \begin{tabular}{|l|}
    \hline
    ~ \vspace{-2.8ex} \\
    #2
    \\
    ~ \vspace{-3ex} \\
    \hline
    ~ \vspace{-2.5ex} \\
    #3
    \\
    ~ \\    % can't vspace here or the line will come out wrong
    \hline
    \end{tabular}
    \hspace{-3ex}
	\end{tabular}
    }
}
\newcommand{\dhgnegdrs}[2]
{
  \mbox{{\huge $\neg$}\hspace{-3.5ex}\dhgdrs{}{#1}{#2}}
}

\newcommand{\dhgboxed}[1]
{
    {
    \begin{tabular}{|l|}
    \hline
    ~ \vspace{-2ex} \\
    #1
    \\
    ~ \\    % can't vspace here or the line will come out wrong
    \hline
    \end{tabular}
    }
}



\begin{document}

\title*{Integrating Logical Representations with\\ Probabilistic Information
using Markov Logic}
% Use \titlerunning{Short Title} for an abbreviated version of
% your contribution title if the original one is too long
\author{Dan Garrette, Katrin Erk, and Raymond Mooney}
% Use \authorrunning{Short Title} for an abbreviated version of
% your contribution title if the original one is too long
\institute{Dan Garrette \at University of Texas at Austin, \email{dhg@cs.utexas.edu}
\and Katrin Erk \at University of Texas at Austin \email{katrin.erk@mail.utexas.edu} 
\and Raymond Mooney \at University of Texas at Austin \email{mooney@cs.utexas.edu}}
%
% Use the package "url.sty" to avoid
% problems with special characters
% used in your e-mail or web address
%
\maketitle


\abstract*{
  First-order logic provides a powerful and flexible mechanism for
  representing natural language semantics.  However, it is an open
  question of how best to integrate it with uncertain, probabilistic
  knowledge, for example regarding word meaning. This paper describes
  the first steps of an approach to recasting first-order semantics
  into the probabilistic models that are part of Statistical
  Relational AI.  Specifically, we show how Discourse Representation
  Structures can be combined with distributional models for word
  meaning inside a Markov Logic Network and used to successfully
  perform inferences that take advantage of logical concepts such as
  factivity as well as probabilistic information on word meaning in
  context.}

\abstract{
  First-order logic provides a powerful and flexible mechanism for
  representing natural language semantics.  However, it is an open
  question of how best to integrate it with uncertain, probabilistic
  knowledge, for example regarding word meaning. This paper describes
  the first steps of an approach to recasting first-order semantics
  into the probabilistic models that are part of Statistical
  Relational AI.  Specifically, we show how Discourse Representation
  Structures can be combined with distributional models for word
  meaning inside a Markov Logic Network and used to successfully
  perform inferences that take advantage of logical concepts such as
  factivity as well as probabilistic information on word meaning in
  context.}



\section{Introduction}

[TODO: Example numbering should be (1) instead of ``Example 1.'']

Logic-based representations of natural language meaning have a long history.  
Representing the meaning of language in a first-order
logical form is appealing because it provides a powerful and flexible way to
express even complex propositions. 
However, systems built solely using first-order logical forms tend to be very
brittle as they have no way of integrating uncertain knowledge. 
They, therefore, tend to have high precision at the cost of low
recall \citep{bos:emnlp2005}.

Recent advances in computational linguistics have yielded robust methods that
use weighted or probabilistic models.  For example, distributional models of
word meaning have been used successfully to judge paraphrase appropriateness.
This has been done by representing the word meaning in context as a point in a
high-dimensional semantics space
\citep{erk:emnlp08,ThaterFuerstenauPinkal:10,erk:acl2010}. However, these
models typically handle only individual phenomena instead of providing a meaning
representation for complete sentences. It is a long-standing open question how
best to integrate the weighted or probabilistic information coming from such
modules with logic-based representations in a way that allows for reasoning over
both.  See, for example, \citet{hobbs:alj93}.

The goal of this work is to combine logic-based meaning representations
with probabilities in a single unified framework.  This will allow us to obtain
the best of both situations: we will have the full expressivity of 
first-order logic and be able to reason with probabilities.  We believe that
this will allow for a more complete and robust approach to natural language
understanding. In order to perform logical inference with probabilities, we draw 
from the large and active body of work related to Statistical Relational AI
\citep{getoor:book2007}.  Specifically, we make use of Markov Logic Networks
(MLNs) \citep{richardson:mlj06} which employ weighted graphical models to
represent first-order logical formulas. MLNs are appropriate for our approach
because they provide an elegant method of assigning weights to first-order
logical rules, combining a diverse set of inference rules, and performing
inference in a probabilistic way. 

While this is a large and complex task, this paper proposes a series
of first steps toward our goal. 
In this paper, we focus on three natural language phenomena and their
interaction: implicativity and factivity, word meaning, and coreference.
Our framework parses natural language
into a logical form, adds rule weights computed by external NLP
modules, and performs inferences using an MLN. Our end-to-end approach
integrates multiple existing tools.
%  within the context of the Natural
% Language Toolkit (NLTK) \citep{bird:book2009}, a Python toolkit for
% natural language processing.  
We use Boxer
\citep{bos:coling2004} to parse natural language into a logical form.
We use Alchemy \citep{kok:tr05} for MLN inference. Finally, we use the
exemplar-based distributional model of \citet{erk:acl2010} to produce
rule weights.

\section{Background}

\textbf{Logic-based semantics.}
Boxer \citep{bos:coling2004} is a software package for wide-coverage semantic
analysis that provides semantic representations in the form of Discourse
Representation Structures \citep{kamp:book93}. It builds on the C\&C CCG parser
\citep{clark:acl04}.
\citet{bos:emnlp2005} describe a system for Recognizing Textual Entailment
(RTE) that uses Boxer to convert both the premise and hypothesis of an RTE pair
into first-order logical semantic representations and then uses a theorem prover
to check for logical entailment. 
% \citet{bos:trec2006} varies this model in order
% to use Boxer in a question answering setting by using Boxer to generate a
% logical representation of a document and a question and attempting to unify the
% two to find an answer to the question.


\noindent\textbf{Distributional models for lexical meaning.}
Distributional models describe the meaning of a word through the context in
which it appears~\citep{landauer97:solution,lund96:producing}, where contexts 
can be documents, other words, or snippets
of syntactic structure. Distributional models are able to predict semantic
similarity between words based on distributional similarity and they can be
learned in an unsupervised fashion. Recently distributional models have been
used to predict the applicability of paraphrases in context
\citep{MitchellLapata:08,erk:emnlp08,ThaterFuerstenauPinkal:10,erk:acl2010}. 
For example,
in ``The wine left a stain'', ``result in'' is a better paraphrase for
``leave'' than is ``entrust'', while the opposite is true in
``He left the children with the nurse''. Usually, the distributional
representation for a word mixes all its usages (senses). For the paraphrase
appropriateness task, these representations are then reweighted, extended, or
filtered to focus on contextually appropriate usages.

\noindent\textbf{Markov Logic.}
An MLN consists of a set of weighted first-order clauses.  It provides a way
of softening first-order logic by making situations in which not all clauses
are satisfied less likely but not impossible \citep{richardson:mlj06}. More
formally, let $X$ be the set of all propositions describing a world (i.e. the
set of all ground atoms), $\mathcal{F}$ be the set of all clauses in the MLN,
$w_i$ be the weight associated with clause $f_i \in \mathcal{F}$,
$\mathcal{G}_{f_i}$ be the set of all possible groundings of clause $f_i$, and
$\mathcal{Z}$ be the normalization constant. Then the probability of a
particular truth assignment $\mathbf{x}$ to the variables in $X$ is defined as:
\begin{equation}
P(X = \mathbf{x}) = \frac{1}{\mathcal{Z}} \exp\left(\sum_{f_i \in \mathcal{F}} w_i \sum_{g \in \mathcal{G}_{f_i}}g(\mathbf{x})
\right) 
 = \frac{1}{\mathcal{Z}} \exp\left(\sum_{f_i \in \mathcal{F}} w_i n_i(\mathbf{x}) \right) \tag{1}\label{e1}
\end{equation}
where $g(\mathbf{x})$ is 1 if $g$ is satisfied and 0 otherwise, and 
$n_i(\mathbf{x})= \sum_{g\in \mathcal{G}_{f_i}}g(\mathbf{x})$ is the number of
groundings of $f_i$ that are satisfied given the current truth assignment to the
variables in $X$. This means that the probability of a truth assignment rises
exponentially with the number of groundings that are satisfied.

Markov Logic has been used previously in other NLP application
(e.g. \citet{poon:emnlp2009}).  However, this paper marks the first attempt at
representing deep logical semantics in an MLN.

While it is possible learn rule weights in an MLN directly from training data,
our approach at this time focuses on incorporating weights computed
by external knowledge sources.  Weights for word meaning rules are computed from
the distributional model of lexical meaning and then injected into the MLN. 
Rules governing implicativity and coreference are given infinite weight
(hard constraints).

% Alchemy \citep{kok:tr05} is an open source software package for MLNs. It
% includes implementations for all of the major existing algorithms for learning and inference with MLNs.

% A Markov Logic Network (MLN) is a probabilistic graphical model used to
% represent weighted formulas of first-order logic.  An MLN represents a set of formulas through a Markov
% network where each vertex stands for a ground atom and each edge
% corresponds to a logical connective.  Each formula results in a
% clique in the network.  Since MLNs only operate on ground atoms, the inference procedure always
% starts by ``grounding out'' all of the formulas to create a complete network. 
% \textbf{KE: The following sentence is not good yet, but I think we should say
% something like this.} In an MLN, a model's goodness grows exponentially worse with the number of formulas that it violates, where the weight on a formula represents the penalty for violating it.

\input{3_data_eval}
\section{Transforming natural language text to logical form}

In transforming natural language text to logical form, we build on the software
package Boxer \citep{bos:coling2004}. Boxer
is an extension to the C\&C parser \citep{clark:acl04} that transforms a parsed
discourse of one or more sentences into a semantic representation.  Boxer
outputs the meaning of each discourse as a Discourse Representation Structure
(DRS) that closely resembles the structures described by \citet{kamp:book93}.

We chose to use Boxer for two main reasons.  First, Boxer is a
wide-coverage system that can deal with arbitrary text. 
%  that is able to return a reasonable logical representation
% of most English sentences.  Since our goal is to work with actual texts,
% it is critical that we have a wide-coverage semantic parser.  If
% we did not, then we would be unable to deal with any but the simplest texts. 
Second, the DRSs that Boxer produces are close to the standard first-order
logical forms that are required for use by the MLN software package 
Alchemy.  Our system transforms Boxer output into a format that Alchemy can read and 
augments it with additional information.

\section{Handling the phenomena}

\subsection*{Implicatives and factives}

\citet{nairn:icos2006} presented an approach to the treatment of inferences
involving implicatives and factives.  Their approach identifies an ``implication
signature'' for every implicative or factive verb that determines the truth
conditions for the verb's nested proposition, whether in a positive or negative
environment.  Implication signatures take the form ``$x/y$'' where $x$
represents the implicativity in the the positive environment and $y$ represents
the implicativity in the negative environment.  Both $x$ and $y$ have three
possible values: ``+'' for positive entailment, meaning the nested proposition
is entailed, ``-'' for negative entailment, meaning the negation of the proposition
is entailed, and ``o'' for ``null'' entailment, meaning that neither the
proposition nor its negation is entailed. Figure \ref{fig:imp-sig} gives
concrete examples.% for two signatures.

% For example, the verb ``managed to'' has positive entailment in the {\it true}
% case and negative entailment under negation.  So, {\it he managed to escape
% $\vDash$ he escaped} while {\it he did not manage to escape $\vDash$ he did not
% escape}.  On the other hand, the verb ``refused to'' has negative entailment
% in the positive case and ``null'' entailment under negation.  So, {\it he
% refused to fight $\vDash$ he did not fight} but {\it he did not refuse to fight}
% entails neither {\it he fought} nor {\it he did not fight}.

\begin{figure}
\begin{center}
  \begin{tabular}{l c l}
    \hline
   	 & ~~~~~~signature~~~~~~ &  \multicolumn{1}{c}{example} \\
   	\hline
   	managed to       & +/- & he managed to escape $\vDash$ he escaped \\
   	                 &     & he did not manage to escape $\vDash$ he did not escape \\
   	\hline
%    	was forced to    & +/o & he was forced to sell $\vDash$ he sold \\
%    	                 &     & he was not forced to sell $?$ he sold \\
%    	\hline
%    	was permitted to & o/- & he was permitted to leave $?$ he left \\
%    	                 &     & he was not permitted to leave $\vDash$ he did not leave \\
%    	\hline
%    	forgot to        & -/+ & he forgot to pay $\vDash$ he did not pay \\
%    	                 &     & he did not forget to pay $\vDash$ he paid \\
%    	\hline
   	refused to       & -/o & he refused to fight $\vDash$ he did not fight \\
   	                 &     & he did not refuse to fight $\nvDash$ \{he fought, he did not fight\} \\
   	\hline
%    	hesitated to     & o/+ & he hesitated to ask $?$ he asked\\
%    	                 &     & he did not hesitate to ask $\vDash$ he asked \\
%    	\hline
%    	admitted that    & +/+ & he admitted that he knew $\vDash$ he knew \\
%    	                 &     & he did not admit that he knew $\vDash$ he knew \\
%    	\hline
%    	pretended that   & -/- & he pretended he was sick $\vDash$ he was not sick \\
%    	                 &     & he did not pretend he was sick $\vDash$ he was not sick\\
%    	\hline
%    	wanted to        & o/o & he wanted to fly $?$ he flew \\  
%    	                 &     & he did not want to fly $?$ he flew \\
%    	\hline
  \end{tabular}
\end{center}
\caption{Implication Signatures}
\label{fig:imp-sig}
\end{figure}

We use these implication signatures to automatically generate rules that
license specific entailments in the MLN.  Since ``forget to'' has implication
signature ``-/+'', we generate the two rules in 
(\ref{ex:imp-fact-rules-managed}).

\begin{small}
\begin{example}\label{ex:imp-fact-rules-managed}
\begin{itemize}
   \item[(a)] $\forall~l_1~l_2~e.[(pred(l_1,``forget",e) \land true(l_1) \land rel(l_1,``theme",e,l_2) \land prop(l_1,l_2)) \rightarrow false(l_2)]]$\footnote{Occurrence-indexing on the predicate ``forget'' has been left out for brevity.}
   \item[(b)] $\forall~l_1~l_2~e.[(pred(l_1,``forget",e) \land false(l_1) \land rel(l_1,``theme",e,l_2) \land prop(l_1,l_2)) \rightarrow true(l_2)]$
\end{itemize}
\end{example}
\end{small}

To understand these rules, consider (\ref{ex:imp-fact-rules-managed}a).  The rule
says that if the atom for the verb ``forget to'' appears in a DRS that has been
determined to be {\it true}, then the DRS representing any ``theme'' proposition
of that verb should be considered {\it false}.  Likewise,
(\ref{ex:imp-fact-rules-managed}b) says that if the occurrence of ``forget to''
appears in a DRS determined to be {\it false}, then the theme DRS should be
considered {\it true}.

Note that when the implication signature indicates a ``null'' entailment, no
rule is generated for that case.  This prevents the MLN from licensing
entailments related directly to the nested proposition, but still allows for
entailments that include the factive verb.  So {\it he wanted to fly} entails
neither {\it he flew} nor {\it he did not fly}, but it does still license {\it
he wanted to fly}.  

% \textbf{KE: The next paragraphs can be cut or shortened severely if needs be.} 
% This approach allows us to capture the consequences of the implication signature
% simply and cleanly with first-order logical rules.  The rules are also generic
% enough to work under various levels of nesting and negation.  Consider again
% (\ref{ex:imp-fact-nested}), which involves nested proposition-introducing verbs
% under negation.  The premise DRS and its flat
% representation are given in Figure \ref{fig:boxer-conversion}.
% The main premise DRS, {\it l0} always {\it true} and the negation modifier ``not''
% entails that the sub-DRS {\it l1} is {\it false}.  Since the verb ``forget to'' is
% positively entailing under negation, the nested DRS {\it l2} is judged as
% {\it true}.  Finally, since the verb ``arrange'' is positively entailing in the
% positive environment, its nested DRS {\it l3} is judged {\it true}. Since the
% ``failing'' act referenced in the hypothesis occurs in a {\it true} DRS, the
% entailment holds.



\subsection*{Ambiguity in word meaning}

In order for our system to be able to make correct natural language inference,
it must be able to handle paraphrasing and deal with hypernymy.  For example,
in order to license the entailment pair in (\ref{ex:syn-hyp-pos}), the system must
recognize that ``owns'' is a valid paraphrase for ``has'', and that ``car'' is a hypernym
of ``convertible''.

\begin{example}\label{ex:syn-hyp-pos}
{\it p:} Ed has a convertible \\
{\it h:} Ed owns a car
\end{example}

In this section we discuss our probabilistic approach to paraphrasing.
In the next section we discuss how this approach is extended to cover
hypernymy. A central problem to solve in the context of paraphrases is
that they are context-dependent. Consider again example
\eqref{ex:prob-wordsense} and its two hypotheses.  The first
hypothesis replaces the word ``sweeping'' with a paraphrase that is
valid in the given context, while the second uses an incorrect
paraphrase. 

To incorporate paraphrasing information into our system, we first
generate rules stating all paraphrase relationships that may
\emph{possibly} apply to a given predicate/hypothesis pair, using
WordNet  \citep{miller:wordnet2009} as a resource. Then we
associate those rules with weights to signal contextual adequacy. For
any two occurrence-indexed words $w_1, w_2$ occurring anywhere in the
premise or hypothesis, we check
whether they co-occur in a WordNet synset. If $w_1, w_2$ have a common synset, 
we generate rules of the form $\forall~l~x.[pred(l,w_1,x) \leftrightarrow
pred(l,w_2,x)]$ to connect them. For named entities, we perform a similar routine:
for each pair of matching named entities found in the premise and hypothesis, we
generate a rule  $\forall~l~x.[named(l,w_1,x) \leftrightarrow named(l,w_2,x)]$.

We then use the distributional model of \citet{erk:acl2010} to compute
paraphrase appropriateness. In the case of \eqref{ex:prob-wordsense}
this means measuring the cosine similarity between the vectors for
``sweep'' and ``cover'' (and between ``sweep'' and ``brush'') in the
sentential context of the premise. MLN formula weights are expected
to be log-odds (i.e., $\log (P/(1-P))$ for some probability $P$), so we
rank all possible paraphrases of a given word $w$ by their cosine similarity
to $w$ and then give them probabilities that decrease by rank according to a
Zipfian distribution.  So, the $k^{th}$ closest paraphrase by cosine similarity
will have probability $P_k$ given by \eqref{eq:zipf}:

\begin{equation}\label{eq:zipf}
P_k \sim 1/k
\end{equation}

% \begin{wrapfigure}{r}{.2\textheight}
% \begin{center}
% \begin{tabular}{lll}
% \hline
%       & P(p) & W(p)   \\
% \hline
% cover &  0.069 & -2.602  \\
% brush &  0.021 & -3.842 \\ 
% \hline
% \end{tabular}
% \end{center}
% \caption{{\small Paraphrase probabilities (left) and log-odds weights (right)}}
% \label{fig:para-weights}
% \end{wrapfigure}

The generated rules are given in \eqref{ex:paraphrase-rules} with the actual
weights calculated for example \eqref{ex:prob-wordsense}.  Note that the
valid paraphrase ``cover'' is given a higher weight than the incorrect 
paraphrase ``brush'', which allows the MLN inference procedure to judge $h_1$ as
a more likely entailment than $h_2$.\footnote{
Because weights are calculated according to the equation $\log(P/(1-P))$, any
paraphrase that has a probability of less than 0.5 will have a negative weight. 
Since most paraphrases will have probabilities less than 0.5, most will yield
negative rule weights.  However, the inferences are still handled properly in
the MLN because the inference is dependent on the {\em relative} weights.}
This same result would not be achieved if we did not take context into
consideration because, without context, ``brush'' is a more likely paraphrase of
``sweep'' than ``cover''.

\begin{example}\label{ex:paraphrase-rules}
\begin{itemize}
  \item[(a)] -2.602~ $\forall~l~x.[pred(l,``v\_sweep\_p\_s0\_w4",x) \leftrightarrow pred(l,``v\_cover\_h\_s0\_w4",x)]$
  \item[(b)] -3.842~ $\forall~l~x.[pred(l,``v\_sweep\_p\_s0\_w4",x) \leftrightarrow pred(l,``v\_brush\_h\_s0\_w4",x)]$
\end{itemize}
\end{example}

Since Alchemy outputs a probability of entailment and not a binary judgment, it
is necessary to specify a probability threshold indicating entailment.  
An appropriate threshold between "entailment" and "non-entailment" will be one
that separates the probability of an inference with the valid rule from the
probability of an inference with the invalid rule.  
While we plan to automatically induce a threshold in the future, our current
implementation uses a value set manually.


\subsection*{Hypernymy}

Like paraphrasehood, hypernymy is context-dependent: In ``A bat flew
out of the cave'', ``animal'' is an appropriate hypernym for ``bat'',
but ``artifact'' is not. So we again use distributional similarity to
determine contextual appropriateness. However, we do not directly
compute cosine similarities between a word and its potential hypernym.
We can hardly assume ``baseball bat'' and ``artifact'' to occur in
similar distributional contexts. So instead of checking for similarity
of ``bat'' and ``artifact'' in a given context, we check ``bat'' and
``club''. That is, we pick a synonym or close hypernym of the word in
question (``bat'') that is also a WordNet hyponym of the hypernym to
check (``artifact'').

A second problem to take into account is the interaction of hypernymy
and polarity. While  \eqref{ex:syn-hyp-pos} is a valid pair, 
\eqref{ex:syn-hyp-neg} is not, because ``have a convertible'' is under
negation. So, we create weighted rules of the form $hypernym(w, h)$,
along with inference rules to guide their interaction with polarity.
We create these rules for all pairs of words $w, h$ in premise and
hypothesis such that $h$ is a hypernym of $w$, again using WordNet to
determine potential hypernymy. 

\begin{example}\label{ex:syn-hyp-neg}
{\it p:} Ed does not have a convertible \\
{\it h:} Ed does not own a car
\end{example}
 
Our inference rules governing the interaction of hypernymy and
polarity are given in (\ref{ex:hypernym-rules}).
The rule in (\ref{ex:hypernym-rules}a) states that in a positive environment,
the hyponym entails the hypernym while the rule in (\ref{ex:hypernym-rules}b)
states that in a negative environment, the opposite is true: the hypernym
entails the hyponym.

\begin{example}\label{ex:hypernym-rules}
\begin{itemize}
  \item[(a)] $\forall~l~p_1~p_2~x.[(hypernym(p_1,p_2) \land true(l) \land pred(l,p_1,x)) \rightarrow pred(l,p_2,x)]]$
  \item[(b)] $\forall~l~p_1~p_2~x.[(hypernym(p_1,p_2) \land false(l) \land pred(l,p_2,x)) \rightarrow pred(l,p_1,x)]]$
\end{itemize}
\end{example}



\subsection*{Making use of coreference information}

As a test case for incorporating additional resources into Boxer's logical form,
we used the coreference data in OntoNotes \citep{hovy:naacl2006}.  However,
the same mechanism would allow us to transfer information into Boxer output from
arbitrary additional NLP tools such as 
automatic coreference analysis tools or semantic role labelers. 
Our system uses coreference information into two distinct ways.

The first way we make use of coreference data is to copy atoms describing a
particular variable to those variables that corefer. 
% Making use of coreference information is often essential in natural language
% inference.  Certain coreferring expressions, however, are more complex. 
Consider again example (\ref{ex:coref}) which has a two-sentence premise. 
This inference requires recognizing that the ``he'' in the second sentence of
the premise refers to ``George Christopher'' from the first sentence.  Boxer
alone is unable to make this connection, but if we receive this information as
input, either from gold-labeled data or a third-party coreference tool, we are
able to incorporate it.  Since Boxer is able to identify the index of the word
that generated a particular predicate, we can tie each predicate to any related
coreference chains.  Then, for each atom on the chain, we can inject copies
of all of the coreferring atoms, replacing the variables to match.
For example, the word ``he'' generates an atom  
{\it pred(l0, male, z5)}\footnote{Atoms simplified for brevity} and
``Christopher'' generates atom {\it named(l0, christopher, x0)}. So, we can
create a new atom by taking the atom for ``christopher'' and replacing the 
label and variable with that of the atom for ``he'',
generating  {\it named(l0, christopher, x5)}.

As a more complex example, the coreference information will inform us that
``the new ballpark'' corefers with ``a replacement for Candlestick Park''.
However, our system is currently unable to handle this coreference correctly at
this time because, unlike the previous example, the expression ``a replacement
for Candlestick Park'' results in a complex three-atom conjunct with two 
separate variables: {\it pred(l2, replacement, x6)}, {\it rel(l2, for, x6, x7)}, and 
{\it named(l2, candlestick\_park, x7)}.  Now, unifying with the atom for ``a
ballpark'', {\it pred(l0, ballpark, x3)}, is not as simple as replacing the
variable because there are two variables to choose from.  Note that it would 
{\it not} be correct to replace
both variables since this would result in a unification of ``ballpark'' with
``candlestick\_park'' which is wrong.  Instead we must determine that {\it x6}
should be the one to unify with {\it x3} while {\it x7} is replaced with a fresh
variable.  The way that we can accomplish this is to look at the dependency
parse of the sentence that is produced by the C\&C parser is a precursor to running
Boxer.  By looking up both ``replacement'' and ``Candlestick Park'' in the
parse, we can determine that ``replacement'' is the head of the phrase, and thus
is the atom whose variable should be unified.  So, we would create new atoms,
{\it pred(l0, replacement, x3)}, {\it rel(l0, for, x3, z0)}, and 
{\it named(l0, candlestick\_park, z0)}, where {\it z0} is a fresh variable.


The second way that we make use of coreference information is to extend the
sentential contexts used for calculating the appropriateness of paraphrases in
the distributional model.  In the simplest case, the sentential context of a
word would simply be the other words in the sentence.  However, consider the
context of the word ``lost'' in the second sentence of \eqref{ex:coref-context}.  

\begin{example}\label{ex:coref-context}
\begin{itemize}
  \item[$p_1$:] In [the final game of the season]$_1$, [the team]$_2$ held on to their lead until overtime
  \item[$p_2$:] But despite that, [they]$_2$ eventually {\bf lost} [it all]$_1$
\end{itemize}
\end{example}

Here we would like to disambiguate ``lost'', but its immediate context, words
like ``despite'' and ``eventually'', gives no indication as to its correct
sense. Our procedure extends the context of the sentence by incorporating all of
the words from all of the phrases that corefer with a word in the immediate
context.  Since coreference chains 1 and 2 have words in $p_2$, the context of
``lost'' ends up including ``final'', ``game'', ``season'', and ``team'' which
give a strong indication of the sense of ``lost''.
Note that using coreference data is stronger than simply expanding the window
because coreferences can cover arbitrarily long distances.

\input{6_evaluation}
\section{Future work}

[pull out realations from the logical form to have more interesting $\alpha$]

[think about vector spaces that take logical form into account similar to how
pado and lapata took dependencies into account.]

[mention vibhav's work]

[mention vectors built form entire sentences.  can measure similarity between
phrases.]



The next step is to execute a full-scale evaluation of our approach using 
more varied phenomena and naturally occurring sentences. 
However, the memory requirements of Alchemy are a limitation that prevents us
from currently executing larger and more complex examples.  The problem arises
because Alchemy considers every possible grounding of every atom, even when a
more focused subset of atoms and inference rules would suffice. There is on-going
work to modify Alchemy so that only the required groundings are incorporated
into the network, reducing the size of the model and thus making it possible to
handle more complex inferences.  We will be able to begin using this new version
of Alchemy very soon and our task will provide an excellent test case for the
modification.

Since Alchemy outputs a probability of entailment, it is necessary to fix a
threshold that separates entailment from nonentailment.
We plan to use machine learning techniques to compute an appropriate threshold
automatically from a calibration dataset such as a corpus of valid and invalid
paraphrases.


\section{Conclusion}

In this paper, we have introduced a system that implements a first step
towards integrating logical semantic representations with
probabilistic weights using methods from Statistical Relational AI,
particularly Markov Logic. We have focused on three phenomena and their
interaction: implicatives, coreference, and word meaning. Taking
implicatives and coreference as categorial and word meaning as
probabilistic, we have used a distributional model to generate
paraphrase appropriateness ratings, which we then transformed into
weights on first order formulas.
The resulting MLN approach is able to correctly solve a number of difficult
textual entailment problems that require handling complex combinations of these
important semantic phenomena.

% The framework we have developed takes a pair of natural
% language sentences as input and parses them into DRS \citep{kamp:book93}
% representations.  It then augments those representations by linking the
% predicates back to the original words in the sentences and incorporating
% coreference information from OntoNotes \citep{hovy:naacl2006}.
% Since DRSs are hierarchical structures, our approach flattens them to a simple
% list of atoms while keeping track of the original structure through DRS labels
% as arguments, allowing inferences to be performed in 
% an MLN. In order to maintain the DRS semantics in the flat logical form, we have 
% hand-written a collection of inference rules that are used in the inference. 
% We also generate a list of inference rules to address the particular
% linguistic phenomena that we are handling.  Categorial rules based
% on implication signatures \citep{nairn:icos2006} are used to handle
% implicativity and factivity.  Weighted rules are used to address issues of
% word meaning in context.



\section*{Acknowledgements}

This work was supported by the Department of Defense (DoD) through a
National Defense Science and Engineering Graduate Fellowship (NDSEG) Fellowship
for the first author, National Science Foundation grant IIS-0845925 for the
second author, and a grant from the Longhorn Innovation Fund for Technology.
[TODO: Is all of this right?]



\bibliographystyle{styles/spbasic}
\bibliography{mln-sem}

\end{document}

